\nonstopmode{}
\documentclass[letterpaper]{book}
\usepackage[times,inconsolata,hyper]{Rd}
\usepackage{makeidx}
\usepackage[utf8]{inputenc} % @SET ENCODING@
% \usepackage{graphicx} % @USE GRAPHICX@
\makeindex{}
\begin{document}
\chapter*{}
\begin{center}
{\textbf{\huge Package `rlibkriging'}}
\par\bigskip{\large \today}
\end{center}
\begin{description}
\raggedright{}
\inputencoding{utf8}
\item[Type]\AsIs{Package}
\item[Title]\AsIs{Kriging model through libKriging binding}
\item[Version]\AsIs{0.1-10}
\item[Date]\AsIs{2020-12-21}
\item[Author]\AsIs{Pascal Havé }\email{hpwxf@haveneer.com}\AsIs{, Yann Richet }\email{yann.richet@irsn.fr}\AsIs{}
\item[Maintainer]\AsIs{Pascal Havé }\email{hpwxf@haveneer.com}\AsIs{}
\item[Description]\AsIs{Binding libKriging to R, and provide DiceKriging features with improved performance.}
\item[License]\AsIs{Apache License (>= 2)}
\item[Encoding]\AsIs{UTF-8}
\item[LinkingTo]\AsIs{Rcpp, RcppArmadillo}
\item[Depends]\AsIs{R (>= 2.14)}
\item[Imports]\AsIs{Rcpp (>= 0.12.11), methods}
\item[Suggests]\AsIs{testthat, DiceKriging, utils}
\item[SystemRequirements]\AsIs{GNU make}
\item[URL]\AsIs{}\url{https://github.com/libKriging}\AsIs{}
\item[RoxygenNote]\AsIs{7.1.1}
\end{description}
\Rdcontents{\R{} topics documented:}
\inputencoding{utf8}
\HeaderA{as.list.Kriging}{List Kriging object content}{as.list.Kriging}
\aliasA{as.list,Kriging,Kriging-method}{as.list.Kriging}{as.list,Kriging,Kriging.Rdash.method}
%
\begin{Description}\relax
List Kriging object content
\end{Description}
%
\begin{Usage}
\begin{verbatim}
## S3 method for class 'Kriging'
as.list(x, ...)
\end{verbatim}
\end{Usage}
%
\begin{Arguments}
\begin{ldescription}
\item[\code{x}] S3 Kriging object

\item[\code{...}] Ignored
\end{ldescription}
\end{Arguments}
%
\begin{Value}
list of Kriging object fields: kernel, optim, objective, theta, sigma2, X, centerX, scaleX, y, centerY, scaleY, regmodel, F, T, M, z, beta
\end{Value}
%
\begin{Author}\relax
Yann Richet (yann.richet@irsn.fr)
\end{Author}
%
\begin{Examples}
\begin{ExampleCode}
f = function(x) 1-1/2*(sin(12*x)/(1+x)+2*cos(7*x)*x^5+0.7)
set.seed(123)
X <- as.matrix(runif(5))
y <- f(X)
r <- Kriging(y, X, "gauss")
l = as.list(r)
cat(paste0(names(l)," =" ,l,collapse="\n"))
\end{ExampleCode}
\end{Examples}
\inputencoding{utf8}
\HeaderA{as\_km}{Build a "as\_km" object, which extends DiceKriging::km S4 class.}{as.Rul.km}
%
\begin{Description}\relax
Build a "as\_km" object, which extends DiceKriging::km S4 class.
\end{Description}
%
\begin{Usage}
\begin{verbatim}
as_km(...)
\end{verbatim}
\end{Usage}
%
\begin{Arguments}
\begin{ldescription}
\item[\code{...}] args
\end{ldescription}
\end{Arguments}
%
\begin{Value}
as\_km/km object
\end{Value}
%
\begin{Author}\relax
Yann Richet (yann.richet@irsn.fr)
\end{Author}
\inputencoding{utf8}
\HeaderA{as\_km.default}{Build a DiceKriging "km" like object.}{as.Rul.km.default}
%
\begin{Description}\relax
Build a DiceKriging "km" like object.
\end{Description}
%
\begin{Usage}
\begin{verbatim}
## Default S3 method:
as_km(
  formula = ~1,
  design,
  response,
  covtype = "matern5_2",
  coef.cov = NULL,
  coef.var = NULL,
  coef.trend = NULL,
  estim.method = "MLE",
  optim.method = "BFGS",
  parinit = NULL,
  ...
)
\end{verbatim}
\end{Usage}
%
\begin{Arguments}
\begin{ldescription}
\item[\code{formula}] R formula object to setup the linear trend (aka Universal Kriging). Supports \textasciitilde{}1, \textasciitilde{}. and \textasciitilde{}.\textasciicircum{}2

\item[\code{design}] data.frame of design of experiments

\item[\code{response}] array of output values

\item[\code{covtype}] covariance structure. Supports "gauss", "exp", ...

\item[\code{coef.cov}] fixed covariance range value (so will not optimize if given)

\item[\code{coef.var}] fixed variance value (so will not estimate if given)

\item[\code{coef.trend}] fixed trend value (so will not estimate if given)

\item[\code{estim.method}] estimation criterion. Supports "MLE" or "LOO"

\item[\code{optim.method}] optimization algorithm used on estim.method objective. Supports "BFGS"

\item[\code{parinit}] initial values of covariance range which will be optimzed using optim.method

\item[\code{...}] Ignored
\end{ldescription}
\end{Arguments}
%
\begin{Value}
as\_km object, extends DiceKriging::km (plus contains a "Kriging" field which contains original object)
\end{Value}
%
\begin{Author}\relax
Yann Richet (yann.richet@irsn.fr)
\end{Author}
%
\begin{Examples}
\begin{ExampleCode}
# a 16-points factorial design, and the corresponding response
d <- 2; n <- 16
design.fact <- expand.grid(x1=seq(0,1,length=4), x2=seq(0,1,length=4))
y <- apply(design.fact, 1, DiceKriging::branin) 

#library(DiceKriging)
# kriging model 1 : matern5_2 covariance structure, no trend, no nugget effect
#m1 <- km(design=design.fact, response=y,covtype = "gauss",parinit = c(.5,1),control = list(trace=F))
as_m1 <- as_km(design=design.fact, response=y,covtype = "gauss",parinit = c(.5,1))
\end{ExampleCode}
\end{Examples}
\inputencoding{utf8}
\HeaderA{as\_km.Kriging}{Convert a "Kriging" object to a DiceKriging::km one.}{as.Rul.km.Kriging}
\aliasA{as\_km,Kriging,Kriging-method}{as\_km.Kriging}{as.Rul.km,Kriging,Kriging.Rdash.method}
%
\begin{Description}\relax
Convert a "Kriging" object to a DiceKriging::km one.
\end{Description}
%
\begin{Usage}
\begin{verbatim}
## S3 method for class 'Kriging'
as_km(k, .call = NULL)
\end{verbatim}
\end{Usage}
%
\begin{Arguments}
\begin{ldescription}
\item[\code{k}] "Kriging" object

\item[\code{.call}] Force the "call" filed in km object
\end{ldescription}
\end{Arguments}
%
\begin{Value}
as\_km object, extends DiceKriging::km plus contains "Kriging" field
\end{Value}
%
\begin{Author}\relax
Yann Richet (yann.richet@irsn.fr)
\end{Author}
%
\begin{Examples}
\begin{ExampleCode}
f = function(x) 1-1/2*(sin(12*x)/(1+x)+2*cos(7*x)*x^5+0.7)
set.seed(123)
X <- as.matrix(runif(5))
y <- f(X)
r <- Kriging(y, X, "gauss")
print(r)
k <- as_km(r)
print(k)
\end{ExampleCode}
\end{Examples}
\inputencoding{utf8}
\HeaderA{Kriging}{Build a "Kriging" object from libKriging.}{Kriging}
%
\begin{Description}\relax
Build a "Kriging" object from libKriging.
\end{Description}
%
\begin{Usage}
\begin{verbatim}
Kriging(
  y,
  X,
  kernel,
  regmodel = "constant",
  normalize = FALSE,
  optim = "BFGS",
  objective = "LL",
  parameters = NULL
)
\end{verbatim}
\end{Usage}
%
\begin{Arguments}
\begin{ldescription}
\item[\code{y}] Array of response values

\item[\code{X}] Matrix of input design

\item[\code{kernel}] Covariance model: "gauss", "exp", ...

\item[\code{regmodel}] Universal Kriging linear trend: "constant", "linear", "interactive" ("constant" by default)

\item[\code{normalize}] Normalize X and y in [0,1] (FALSE by default)

\item[\code{optim}] Optimization method to fit hyper-parameters: "BFGS", "Newton" (uses objective Hessian), "none" (keep initial "parameters" values)

\item[\code{objective}] Objective function to optimize: "LL" (log-Likelihood, by default), "LOO" (leave one out)

\item[\code{parameters}] Initial hyper parameters: list(sigma2=..., theta=...). If theta has many rows, each is used as a starting point for optim.
\end{ldescription}
\end{Arguments}
%
\begin{Value}
S3 Kriging object. Should be used with its predict, simulate, update methods.
\end{Value}
%
\begin{Author}\relax
Yann Richet (yann.richet@irsn.fr)
\end{Author}
\inputencoding{utf8}
\HeaderA{leaveOneOut}{Compute model leave-One-Out error at given args}{leaveOneOut}
%
\begin{Description}\relax
Compute model leave-One-Out error at given args
\end{Description}
%
\begin{Usage}
\begin{verbatim}
leaveOneOut(...)
\end{verbatim}
\end{Usage}
%
\begin{Arguments}
\begin{ldescription}
\item[\code{...}] args
\end{ldescription}
\end{Arguments}
%
\begin{Value}
leave-One-Out
\end{Value}
\inputencoding{utf8}
\HeaderA{leaveOneOut.Kriging}{Compute leave-One-Out of Kriging model}{leaveOneOut.Kriging}
\aliasA{leaveOneOut,Kriging,Kriging-method}{leaveOneOut.Kriging}{leaveOneOut,Kriging,Kriging.Rdash.method}
%
\begin{Description}\relax
Compute leave-One-Out of Kriging model
\end{Description}
%
\begin{Usage}
\begin{verbatim}
## S3 method for class 'Kriging'
leaveOneOut(object, theta, grad = FALSE)
\end{verbatim}
\end{Usage}
%
\begin{Arguments}
\begin{ldescription}
\item[\code{object}] S3 Kriging object

\item[\code{theta}] new points in model output space

\item[\code{grad}] return Gradient ? (default is TRUE)
\end{ldescription}
\end{Arguments}
%
\begin{Value}
leave-One-Out computed for given theta
\end{Value}
%
\begin{Author}\relax
Yann Richet (yann.richet@irsn.fr)
\end{Author}
%
\begin{Examples}
\begin{ExampleCode}
f = function(x) 1-1/2*(sin(12*x)/(1+x)+2*cos(7*x)*x^5+0.7)
set.seed(123)
X <- as.matrix(runif(5))
y <- f(X)
r <- Kriging(y, X, "gauss",objective="LOO")
print(r)
loo = function(theta) leaveOneOut(r,theta)$leaveOneOut
t = seq(0.0001,2,,101)
  plot(t,loo(t),type='l')
  abline(v=as.list(r)$theta,col='blue')
\end{ExampleCode}
\end{Examples}
\inputencoding{utf8}
\HeaderA{logLikelihood}{Compute model log-Likelihood at given args}{logLikelihood}
%
\begin{Description}\relax
Compute model log-Likelihood at given args
\end{Description}
%
\begin{Usage}
\begin{verbatim}
logLikelihood(...)
\end{verbatim}
\end{Usage}
%
\begin{Arguments}
\begin{ldescription}
\item[\code{...}] args
\end{ldescription}
\end{Arguments}
%
\begin{Value}
log-Likelihood
\end{Value}
\inputencoding{utf8}
\HeaderA{logLikelihood.Kriging}{Compute log-Likelihood of Kriging model}{logLikelihood.Kriging}
\aliasA{logLikelihood,Kriging,Kriging-method}{logLikelihood.Kriging}{logLikelihood,Kriging,Kriging.Rdash.method}
%
\begin{Description}\relax
Compute log-Likelihood of Kriging model
\end{Description}
%
\begin{Usage}
\begin{verbatim}
## S3 method for class 'Kriging'
logLikelihood(object, theta, grad = FALSE, hess = FALSE)
\end{verbatim}
\end{Usage}
%
\begin{Arguments}
\begin{ldescription}
\item[\code{object}] S3 Kriging object

\item[\code{theta}] new points in model output space

\item[\code{grad}] return Gradient ? (default is TRUE)

\item[\code{hess}] return Hessian ? (default is FALSe)
\end{ldescription}
\end{Arguments}
%
\begin{Value}
log-Likelihood computed for given theta
\end{Value}
%
\begin{Author}\relax
Yann Richet (yann.richet@irsn.fr)
\end{Author}
%
\begin{Examples}
\begin{ExampleCode}
f = function(x) 1-1/2*(sin(12*x)/(1+x)+2*cos(7*x)*x^5+0.7)
set.seed(123)
X <- as.matrix(runif(5))
y <- f(X)
r <- Kriging(y, X, "gauss")
print(r)
ll = function(theta) logLikelihood(r,theta)$logLikelihood
t = seq(0.0001,2,,101)
  plot(t,ll(t),type='l')
  abline(v=as.list(r)$theta,col='blue')
\end{ExampleCode}
\end{Examples}
\inputencoding{utf8}
\HeaderA{predict.as\_km}{Overload DiceKriging::predict.km for as\_km objects (expected faster).}{predict.as.Rul.km}
\aliasA{predict,as\_km,as\_km-method}{predict.as\_km}{predict,as.Rul.km,as.Rul.km.Rdash.method}
%
\begin{Description}\relax
Overload DiceKriging::predict.km for as\_km objects (expected faster).
\end{Description}
%
\begin{Usage}
\begin{verbatim}
## S3 method for class 'as_km'
predict(
  object,
  newdata,
  type = "UK",
  se.compute = TRUE,
  cov.compute = FALSE,
  light.return = TRUE,
  bias.correct = FALSE,
  checkNames = FALSE,
  ...
)
\end{verbatim}
\end{Usage}
%
\begin{Arguments}
\begin{ldescription}
\item[\code{object}] as\_km object

\item[\code{newdata}] matrix of points where to perform prediction

\item[\code{type}] kriging family ("UK")

\item[\code{se.compute}] compute standard error (TRUE by default)

\item[\code{cov.compute}] compute covariance matrix between newdata points (FALSE by default)

\item[\code{light.return}] return no other intermediate objects (like T matrix) (default is TRUE)

\item[\code{bias.correct}] fix UK variance and covaariance (defualt is FALSE)

\item[\code{checkNames}] check consistency between object design data: X and newdata (default is FALSE)

\item[\code{...}] Ignored
\end{ldescription}
\end{Arguments}
%
\begin{Value}
list of predict data: mean, sd, trend, cov, upper95 and lower95 quantiles.
\end{Value}
%
\begin{Author}\relax
Yann Richet (yann.richet@irsn.fr)
\end{Author}
%
\begin{Examples}
\begin{ExampleCode}
# a 16-points factorial design, and the corresponding response
d <- 2; n <- 16
design.fact <- expand.grid(x1=seq(0,1,length=4), x2=seq(0,1,length=4))
y <- apply(design.fact, 1, DiceKriging::branin) 

#library(DiceKriging)
# kriging model 1 : matern5_2 covariance structure, no trend, no nugget effect
#m1 <-      km(design=design.fact, response=y,covtype = "gauss",parinit = c(.5,1),control = list(trace=F))
as_m1 <- as_km(design=design.fact, response=y,covtype = "gauss",parinit = c(.5,1))
as_p = predict(as_m1,newdata=matrix(.5,ncol=2),type="UK",checkNames=FALSE,light.return=TRUE)
\end{ExampleCode}
\end{Examples}
\inputencoding{utf8}
\HeaderA{predict.Kriging}{Predict Kriging model at given points}{predict.Kriging}
\aliasA{predict,Kriging,Kriging-method}{predict.Kriging}{predict,Kriging,Kriging.Rdash.method}
%
\begin{Description}\relax
Predict Kriging model at given points
\end{Description}
%
\begin{Usage}
\begin{verbatim}
## S3 method for class 'Kriging'
predict(object, x, stdev = T, cov = F, ...)
\end{verbatim}
\end{Usage}
%
\begin{Arguments}
\begin{ldescription}
\item[\code{object}] S3 Kriging object

\item[\code{x}] points in model input space where to predict

\item[\code{stdev}] return also standard deviation (default TRUE)

\item[\code{cov}] return covariance matrix between x points (default FALSE)

\item[\code{...}] Ignored
\end{ldescription}
\end{Arguments}
%
\begin{Value}
list containing: mean, stdev, cov
\end{Value}
%
\begin{Author}\relax
Yann Richet (yann.richet@irsn.fr)
\end{Author}
%
\begin{Examples}
\begin{ExampleCode}
f = function(x) 1-1/2*(sin(12*x)/(1+x)+2*cos(7*x)*x^5+0.7)
  plot(f)
set.seed(123)
X <- as.matrix(runif(5))
y <- f(X)
  points(X,y,col='blue')
r <- Kriging(y, X, "gauss")
x = seq(0,1,,101)
p_x = predict(r, x)
  lines(x,p_x$mean,col='blue')
  lines(x,p_x$mean-2*p_x$stdev,col='blue')
  lines(x,p_x$mean+2*p_x$stdev,col='blue')
\end{ExampleCode}
\end{Examples}
\inputencoding{utf8}
\HeaderA{print.Kriging}{Print Kriging object content}{print.Kriging}
\aliasA{print,Kriging,Kriging-method}{print.Kriging}{print,Kriging,Kriging.Rdash.method}
%
\begin{Description}\relax
Print Kriging object content
\end{Description}
%
\begin{Usage}
\begin{verbatim}
## S3 method for class 'Kriging'
print(x, ...)
\end{verbatim}
\end{Usage}
%
\begin{Arguments}
\begin{ldescription}
\item[\code{x}] S3 Kriging object

\item[\code{...}] Ignored
\end{ldescription}
\end{Arguments}
%
\begin{Author}\relax
Yann Richet (yann.richet@irsn.fr)
\end{Author}
%
\begin{Examples}
\begin{ExampleCode}
f = function(x) 1-1/2*(sin(12*x)/(1+x)+2*cos(7*x)*x^5+0.7)
set.seed(123)
X <- as.matrix(runif(5))
y <- f(X)
r <- Kriging(y, X, "gauss")
print(r)
\end{ExampleCode}
\end{Examples}
\inputencoding{utf8}
\HeaderA{simulate.as\_km}{Overload DiceKriging::simulate.km for as\_km objects (expected faster).}{simulate.as.Rul.km}
\aliasA{simulate,as\_km,as\_km-method}{simulate.as\_km}{simulate,as.Rul.km,as.Rul.km.Rdash.method}
%
\begin{Description}\relax
Overload DiceKriging::simulate.km for as\_km objects (expected faster).
\end{Description}
%
\begin{Usage}
\begin{verbatim}
## S3 method for class 'as_km'
simulate(
  object,
  nsim = 1,
  seed = NULL,
  newdata,
  cond = TRUE,
  nugget.sim = 0,
  checkNames = FALSE,
  ...
)
\end{verbatim}
\end{Usage}
%
\begin{Arguments}
\begin{ldescription}
\item[\code{object}] as\_km object

\item[\code{nsim}] number of response vector to simulate

\item[\code{seed}] random seed

\item[\code{newdata}] matrix of points where to perform prediction

\item[\code{cond}] simulate conditional samples (only TRUE accepted)

\item[\code{nugget.sim}] numercial ngget ,effect to avoid numerical unstabilities

\item[\code{checkNames}] check consistency between object design data: X and newdata (default is FALSE)

\item[\code{...}] Ignored
\end{ldescription}
\end{Arguments}
%
\begin{Value}
length(x) x nsim matrix containing simulated path at newdata points
\end{Value}
%
\begin{Author}\relax
Yann Richet (yann.richet@irsn.fr)
\end{Author}
%
\begin{Examples}
\begin{ExampleCode}
f = function(x) 1-1/2*(sin(12*x)/(1+x)+2*cos(7*x)*x^5+0.7)
  plot(f)
set.seed(123)
X <- as.matrix(runif(5))
y <- f(X)
  points(X,y,col='blue')
k <- as_km(design=X, response=y,covtype = "gauss")
x = seq(0,1,,101)
s_x = simulate(k, nsim=3, newdata=x)
  lines(x,s_x[,1],col='blue')
  lines(x,s_x[,2],col='blue')
  lines(x,s_x[,3],col='blue')
\end{ExampleCode}
\end{Examples}
\inputencoding{utf8}
\HeaderA{simulate.Kriging}{Simulate (conditional) Kriging model at given points}{simulate.Kriging}
\aliasA{simulate,Kriging,Kriging-method}{simulate.Kriging}{simulate,Kriging,Kriging.Rdash.method}
%
\begin{Description}\relax
Simulate (conditional) Kriging model at given points
\end{Description}
%
\begin{Usage}
\begin{verbatim}
## S3 method for class 'Kriging'
simulate(object, nsim = 1, seed = 123, x, ...)
\end{verbatim}
\end{Usage}
%
\begin{Arguments}
\begin{ldescription}
\item[\code{object}] S3 Kriging object

\item[\code{nsim}] number of simulations to perform

\item[\code{seed}] random seed used

\item[\code{x}] points in model input space where to simulate

\item[\code{...}] Ignored
\end{ldescription}
\end{Arguments}
%
\begin{Value}
length(x) x nsim matrix containing simulated path at x points
\end{Value}
%
\begin{Author}\relax
Yann Richet (yann.richet@irsn.fr)
\end{Author}
%
\begin{Examples}
\begin{ExampleCode}
f = function(x) 1-1/2*(sin(12*x)/(1+x)+2*cos(7*x)*x^5+0.7)
  plot(f)
set.seed(123)
X <- as.matrix(runif(5))
y <- f(X)
  points(X,y,col='blue')
r <- Kriging(y, X, "gauss")
x = seq(0,1,,101)
s_x = simulate(r, nsim=3, x=x)
  lines(x,s_x[,1],col='blue')
  lines(x,s_x[,2],col='blue')
  lines(x,s_x[,3],col='blue')
\end{ExampleCode}
\end{Examples}
\inputencoding{utf8}
\HeaderA{update.as\_km}{Overload DiceKriging::update.km methd for as\_km objects (expected faster).}{update.as.Rul.km}
\aliasA{update,as\_km,as\_km-method}{update.as\_km}{update,as.Rul.km,as.Rul.km.Rdash.method}
%
\begin{Description}\relax
Overload DiceKriging::update.km methd for as\_km objects (expected faster).
\end{Description}
%
\begin{Usage}
\begin{verbatim}
## S3 method for class 'as_km'
update(
  object,
  newX,
  newy,
  newX.alreadyExist = FALSE,
  cov.reestim = TRUE,
  trend.reestim = cov.reestim,
  nugget.reestim = FALSE,
  newnoise.var = NULL,
  kmcontrol = NULL,
  newF = NULL,
  ...
)
\end{verbatim}
\end{Usage}
%
\begin{Arguments}
\begin{ldescription}
\item[\code{object}] as\_km object

\item[\code{newX}] new design points: matrix of object@d columns

\item[\code{newy}] new response points

\item[\code{newX.alreadyExist}] if TRUE, newX contains some ppoints already in object@X

\item[\code{cov.reestim}] fit object to newdata: estimate theta (only supports TRUE)

\item[\code{trend.reestim}] fit object to newdata: estimate beta (only supports TRUE)

\item[\code{nugget.reestim}] fit object to newdata: estimate nugget effect (only support FALSE)

\item[\code{newnoise.var}] add noise to newy response

\item[\code{kmcontrol}] parametrize fit (unsupported)

\item[\code{newF}] 

\item[\code{...}] Ignored
\end{ldescription}
\end{Arguments}
%
\begin{Author}\relax
Yann Richet (yann.richet@irsn.fr)
\end{Author}
%
\begin{Examples}
\begin{ExampleCode}
f = function(x) 1-1/2*(sin(12*x)/(1+x)+2*cos(7*x)*x^5+0.7)
  plot(f)
set.seed(123)
X <- as.matrix(runif(5))
y <- f(X)
  points(X,y,col='blue')
k <- as_km(design=X, response=y,covtype = "gauss")
x = seq(0,1,,101)
p_x = predict(k, x)
  lines(x,p_x$mean,col='blue')
  lines(x,p_x$lower95,col='blue')
  lines(x,p_x$upper95,col='blue')
newX <- as.matrix(runif(3))
newy <- f(newX)
  points(newX,newy,col='red')
update(k,newy,newX)
x = seq(0,1,,101)
p2_x = predict(k, x)
  lines(x,p2_x$mean,col='red')
  lines(x,p2_x$lower95,col='red')
  lines(x,p2_x$upper95,col='red')
\end{ExampleCode}
\end{Examples}
\inputencoding{utf8}
\HeaderA{update.Kriging}{Update Kriging model with new points}{update.Kriging}
\aliasA{update,Kriging,Kriging-method}{update.Kriging}{update,Kriging,Kriging.Rdash.method}
%
\begin{Description}\relax
Update Kriging model with new points
\end{Description}
%
\begin{Usage}
\begin{verbatim}
## S3 method for class 'Kriging'
update(object, newy, newX, normalize = FALSE, ...)
\end{verbatim}
\end{Usage}
%
\begin{Arguments}
\begin{ldescription}
\item[\code{object}] S3 Kriging object

\item[\code{newy}] new points in model output space

\item[\code{newX}] new points in model input space

\item[\code{normalize}] Normalize X and y in [0,1] (FALSE by default)

\item[\code{...}] Ignored
\end{ldescription}
\end{Arguments}
%
\begin{Author}\relax
Yann Richet (yann.richet@irsn.fr)
\end{Author}
%
\begin{Examples}
\begin{ExampleCode}
f = function(x) 1-1/2*(sin(12*x)/(1+x)+2*cos(7*x)*x^5+0.7)
  plot(f)
set.seed(123)
X <- as.matrix(runif(5))
y <- f(X)
  points(X,y,col='blue')
r <- Kriging(y, X, "gauss")
x = seq(0,1,,101)
p_x = predict(r, x)
  lines(x,p_x$mean,col='blue')
  lines(x,p_x$mean-2*p_x$stdev,col='blue')
  lines(x,p_x$mean+2*p_x$stdev,col='blue')
newX <- as.matrix(runif(3))
newy <- f(newX)
  points(newX,newy,col='red')
update(r,newy,newX)
x = seq(0,1,,101)
p2_x = predict(r, x)
  lines(x,p2_x$mean,col='red')
  lines(x,p2_x$mean-2*p2_x$stdev,col='red')
  lines(x,p2_x$mean+2*p2_x$stdev,col='red')
\end{ExampleCode}
\end{Examples}
\printindex{}
\end{document}
